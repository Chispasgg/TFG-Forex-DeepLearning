Este proyecto se ha dividido en m'odulos. Este planteamiento nos permitir'a cerrar bloques completos para poder avanzar en el desarrollo sin necesidad de volver atr'as a modificar/corregir los m'odulos ya finalizados.

Este trabajo ha sido desarrollado por dos alumnos, por ello todas las responsabilidades se han dividido entre ambos. Se acord'o, al principio del proyecto, que ambos alumnos estar'ian presentes en el desarrollo todos los m'odulos, con el objetivo de estar los dos al tanto de cualquier problema que surgiera en cada uno de ellos.

Adem'as, se han definido unos roles para el desarrollo de las tareas. De esta forma, nos adaptaremos a la mentalidad que exige el rol para el desempeño de la tarea.
\\

Estos roles son:
\begin{itemize}
\item \textbf{Jefe de proyecto}: Puesto de mayor responsabilidad, en el que hay tomar decisiones que puedan cambiar el rumbo del proyecto.
\item \textbf{Analista}: Las tareas del analista ser'an tareas para definir otras tareas, y para mostrar al cliente el resultado del trabajo.
\item \textbf{Cient'ifico de datos}: Llevar'a a cabo todas las tareas relacionadas con los algoritmos de Deep Learning.
\item \textbf{Desarrollador}: Llevar'a a cabo las tareas de desarrollado e implementaci'on del sistema.
\end{itemize}

Los m'odulos en los que se ha dividido este proyecto son:

\begin{enumerate}
\item \textbf{Inicio del proyecto} - Jefe de proyecto y Analista\\
Este m'odulo tiene como fin planificar el desarrollo del proyecto. Se llevar'an a cabo diferentes reuniones entre los miembros del equipo y el profesor para realizar una planificaci'on realista con un alcance alto.\\

\clearpage

Se realizar'an las siguientes tareas:
\begin{itemize}
\item Identificaci'on de objetivos, alcances, fortalezas, problemas.
\item Estudio previo del problema.
\item Diseño de la implementaci'on.
\item Elecci'on de tecnolog'ias.
\item Brain storming para conseguir ideas sobre diseños de arquitecturas.
\end{itemize}


\item \textbf{Requisitos} - Analista\\
Durante este proceso se realizar'an reuniones para elaborar los requisitos en base a las tecnolog'ias a utilizar. Es una fase crucial en el proyecto, ya que a partir de este m'odulo se empieza el desarollo.\\
Se realizar'as las siguientes tareas:
\begin{itemize}
\item Elicitaci'on de requisitos.
\item Documentaci'on de estos requisitos.
\end{itemize}


\item \textbf{Estudio Formativo }- Analista, Cient'ifico de datos y Desarrollador\\
Se han elegido tecnolog'ias muy innovadoras para el desarrollo de este proyecto. Estas tecnolog'ias requieren de un conocimiento previo para poder desarrollar los m'odulos de forma eficiente.\\
Durante esta fase se estudiar'an:
\begin{itemize}
\item Mercado Forex.
\item Sistemas Big Data, Apache Spark, Apache Kafka.
\item Fundamentos te'oricos del Deep Learning y herramientas para poder desarrollar estos algoritmos, como Tensorflow.
\end{itemize}


\item \textbf{Extracci'on de datos hist'oricos} - Analista\\
Este m'odulo tiene como objetivo obtener los datos de las monedas en el proyecto para usarlos durante el desarrollo. Estos datos tienen que cumplir unos requisitos m'inimos, como la calidad del dato (6 decimales), y el nivel de detalle (datos a nivel de minuto).


\item \textbf{M'odulo Extracci'on de informaci'on} - Analista y Desarrollador\\
En este m'odulo se definir'an y desarrollar'an los extractores de datos. Estos datos se deben recoger en tiempo real.\\
Las tareas a desarrollar son:
\begin{itemize}
\item Aprendizaje sobre el framework Beautifulsoup.
\item Diseño de un sistema escalable.
\item Programaci'on de los scripts de los extractores.
\item Pruebas del sistema.
\end{itemize}


\item \textbf{M'odulo Almacenamiento de informaci'on} - Analista y Desarrollador\\
En este m'odulo se configurar'an y desarrollar'an los sistemas encargados de almacenar la informaci'on. Esta informaci'on es la informaci'on extraída en tiempo real, la informaci'on procesada y la informaci'on que el modelo va a generar.\\
Las tareas a desarrollar son:
\begin{itemize}
\item Estudio del funcionamiento de Apache Kafka, y las diferentes formas de implementarlo.
\item Instalaci'on de las herramientas Apache Spark y Apache Kafka.
\item Implementar la conexi'on entre Spark y Kafka.
\item Optimizaci'on del sistema Apache Spark.
\item Dockerizar el entorno.
\end{itemize}



\item \textbf{M'odulo Red Neuronal} - Analista y Cient'ifico de datos\\
Durante esta fase se desarrollarán los modelos predictivos basados en redes neuronales.\\
Las tareas a desarrollar son:
\begin{itemize}
\item Diseñar la arquitectura de la red neuronal recurrente y convolucional.
\item Búsqueda y selecci'on de los hiperpar'ametros de las redes neuronales.
\item Pruebas del sistema.
\item Selecci'on del formato de los datos y de la salida de las redes neuronales.
\end{itemize}


\item \textbf{M'odulo Capa de presentaci'on} - Jefe de proyecto, Analista y Desarrollador\\
Se desarrollar'a un sistema front-end para visualizar los datos que ha procesado la red neuronal.\\ Las tareas a desarrollar son:
\begin{itemize}
\item Investigaci'on de las aplicaciones web reactivas.
\item Diseño y evaluaci'on de la interfaz.
\item Dockerizar el entorno del m'odulo.
\item Integrar este sistema con el resto.
\end{itemize}


\item \textbf{Experimentaci'on} - Analista, Cient'ifico de datos y Desarrollador\\
En este m'odulo, se realizar'an pruebas de todos los sistemas desarrollados, poniendo especial enf'asis en las redes neuronales.
\begin{itemize}
\item Se evaluar'an las redes con monedas de baja volatilidad y alta volatilidad, junto con los resultados de ambas redes.
\item Se evaluar'a el funcionamiento conjunto de los sistemas.
\item Se evaluar'an las arquitecturas de las redes para encontrar posibles fallos de rendimiento
\end{itemize}


\item \textbf{Memoria} - Jefe de proyecto\\
Se elaborar'a la documentaci'on de todos los conocimientos y resultados obtenidos durante el periodo de desarrollo.

\item \textbf{Defensa} - Jefe de proyecto\\
Realizaci'on de la defensa del Trabajo de Fin de Grado.
\end{enumerate}

\clearpage