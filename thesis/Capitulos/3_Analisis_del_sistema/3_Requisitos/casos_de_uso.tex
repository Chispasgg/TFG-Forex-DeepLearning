Los casos de uso que recoge la aplicaci'on se detallan a continuaci'on:

\cuadro{|c|p{0.8\linewidth}|}{Caso de uso 001}{caso_de_uso_001}{
\rowcolor[HTML]{CBCEFB} \hline \textbf{CU-001} & \textbf{Visualizaci'on de predicciones}
\\ \hline \textbf{Descripci'on}  & El sistema deber'a comportarse tal como se describe en el siguiente caso de uso cuando el usuario desee ver las predicciones generadas por el sistema.
\\ \hline \textbf{Dependencias} & 
\begin{itemize}
\item OBJ-003
\item RF-003
\end{itemize}
\\ \hline \textbf{Precondici'on} & El sistema ha entrenado correctamente y existen resultados en la base de datos.
\\ \hline \textbf{Secuencia normal}  &
\begin{enumerate}
\item El usuario accede a la interfaz web.
\item El sistema renderiza una gráfica con los valores de la divisa EUR/USD para la red recurrente como valores por defecto.
\item El usuario, mediante el men'u desplegable, elije una divisa.
\item El sistema renderiza una nueva gŕafica para la divisa seleccionada.
\item El usuario, mediante las pestañas, cambia de tipo red.
\item El sistema renderiza una nueva gr'afica para el tipo de red seleccionado.
\end{enumerate}    
\\}
\clearpage

\cuadro{|c|p{0.8\linewidth}|}{Caso de uso 002}{caso_de_uso_002}{
\rowcolor[HTML]{CBCEFB} \hline \textbf{CU-002} & \textbf{Visualizaci'on de metadatos}
\\ \hline \textbf{Descripci'on}  & El sistema deber'a comportarse tal como se describe en el siguiente caso de uso cuando el usuario desee ver los metadatos.
\\ \hline \textbf{Dependencias} & 
\begin{itemize}
\item OBJ-003
\item RF-005
\end{itemize}
\\ \hline \textbf{Precondici'on} & El sistema ha entrenado correctamente y existen metadatos en la base de datos.
\\ \hline \textbf{Secuencia normal}  &
\begin{enumerate}
\item El usuario accede a la interfaz web.
\item El sistema renderiza una tabla con los metadatos de la divisa EUR/USD para la red recurrente como valores por defecto.
\item El usuario, mediante el men'u desplegable, elije una divisa.
\item El sistema renderiza una nueva tabla para la divisa seleccionada.
\item El usuario, mediante las pestañas, cambia de tipo red.
\item El sistema renderiza una nueva tabla para el tipo de red seleccionado.
\end{enumerate}    
\\}
\clearpage


\cuadro{|c|p{0.8\linewidth}|}{Caso de uso 003}{caso_de_uso_003}{
\rowcolor[HTML]{CBCEFB} \hline \textbf{CU-003} & \textbf{Interacci'on con la gr'afica}
\\ \hline \textbf{Descripci'on}  & El sistema deber'a comportarse tal como se describe en el siguiente caso de uso cuando el usuario desee interactuar con la gr'afica de resultados.
\\ \hline \textbf{Dependencias} & 
\begin{itemize}
\item OBJ-003
\item RF-005
\item RI-003
\end{itemize}
\\ \hline \textbf{Precondici'on} & El sistema ha entrenado correctamente y existen resultados en la base de datos.
\\ \hline \textbf{Secuencia normal}  &
\begin{enumerate}
\item El usuario accede a la interfaz web.
\item El sistema renderiza una tabla con los metadatos de la divisa EUR/USD para la red recurrente como valores por defecto.
\item El usuario accede a la interfaz web.
\item El sistema renderiza una tabla con los metadatos de la divisa EUR/USD para la red recurrente como valores por defecto.
\item El usuario, mediante el menú en la esquina superior derecha de la gráfica, aumenta el zoom de la m'isma.
\item El usuario, mediante el menú en la esquina superior derecha de la gráfica, disminuye el zoom de la misma.
\end{enumerate}    
\\}
\clearpage