% !TEX root = ../proyect.tex

\chapter{Diseño del sistema}\label{cap4}

El objetivo principal del sistema es proveer un modelo de aprendizaje profundo que genere predicciones sobre el mercado Forex. Antes de entrar en cuestiones de diseño, tenemos que hablar acerca del flujo lógico por el que debe pasar la información.

\figura{1}{img/disenio_sistema/disenio2.png}{Diagrama de diseño de flujo de la informaci'on}{disenio2}{}

En la figura \ref{disenio2} se ilustra del diseño el sistema y el flujo de la información a través del mismo.
\begin{itemize}
\item La información es generada por los productores, uno por divisa, la cual es enviada al gestor de cola de mensajes. 
\item El gestor almacena temporalmente todos los mensajes recibidos por los productores hasta que son requeridos. 
\item La información viaja al módulo de procesamiento, donde se realizan las transformaciones necesarias sobre los datos. 
\item Una vez se tienen los datos tratados se almacenan en una base de datos donde se ponen a disposición de la redes neuronales. 
\item La redes toman los datos para su entrenamiento y generan las predicciones. Se genera una predicción para los siguientes cuarenta y cinco minutos por red y divisa. 
\item Las predicciones se almacenan en la base de datos donde una interfaz de usuario web puede acceder a ellas y pueden ser representadas.
\end{itemize}


\pagebreak


Una vez visto el flujo del sistema, se ha agrupado las tareas del sistema en cuatro grupos, estos son: 
\begin{itemize}
\item \textbf{Obtención}: Se necesitan datos históricos de las divisas para poder entrenar el modelo.
\item \textbf{Tratamiento}: La información obtenida necesita ser tratada antes de que el modelo pueda aprender de ella.
\item \textbf{Almacenamiento}: Como cualquier sistema de información se necesita un sistema de almacenamiento permanente.
\item \textbf{Construcción del modelo}: El desarrollo y entrenamiento del modelo.
\item \textbf{Visualización}: Los datos generados por el modelo se deben poder representar correctamente para su monitorización e interpretación.
\end{itemize}


Para resolver todos estos problemas se ha diseñado una arquitectura modular en la que cada módulo resuelve una de las problemáticas vistas anteriormente. Se ha optado por esta arquitectura ya que el desacoplamiento de los componentes permiten, por una parte, un diseño, implementación y testeo sencillos y por, otra parte, permite aumentar el número de instancias en ejecución de cada módulo en caso de necesidad, usando tecnología de contenedores. Los módulos que componen el sistema son:
\begin{itemize}
\item Productores de datos
\item Procesado y almacenamiento
\item Red neuronal
\item Monitorización
\end{itemize}

Todos ellos serán explicados en este capítulo.


\pagebreak

\section{Tecnolog'ias troncales}
Existen algunas tecnologías comunes a todo el desarrollo, las cuales se encuentran en menor o mayor medida en todo el sistema, las cuales son las tecnologías troncales.

\subsection{Python}
Python es un lenguaje de programación interpretado multiparadigma y multiplataforma de tipado dinámico.

Python junto con R son los lenguajes predominantes en el campo del Machine Learning y el Deep Learning. En el caso de Python, su éxito se debe a varios factores, como son su sintaxis sencilla, la cual favorece un código legible y un desarrollo ágil, facilidad para adaptar librerías de lenguajes de más bajo nivel, como son C, C++ o CUDA, los cuales tienen un desempeño mejor pero no es viable implementar la totalidad de los modelos con ello y por una gran variedad de librerias y frameworks open soruce. También hay que tener en cuenta el apoyo de grandes empresas como Google, con su framework TensorFlow, siendo uno de los frameworks más utilizados para Deep Learning.

Se ha elegido este lenguaje tanto por las razones previamente presentadas como porque, al ser un lenguaje multiparadigma usado en campos tan variados como el desarrollo web, scripting, machine learning, ciberseguridad, matemáticas, fisica, etc., nos permite implementar la totalidad del sistema con él, permitiendo una integración más sencilla de los componentes.


\subsection{Docker}
Docker es un proyecto para la automatización y el despliegue de aplicaciones dentro de contenedores de software que proporciona una capa extra de abstracci'on y virtualizaci'on de aplicaciones en m'ultiples sistemas operativos. Gracias a docker no es necesario virtualizar sistemas operativos completos, como ocurre con las m'aquinas virtuales, sino que se pueden virtualizar únicamente los componentes que sean necesarios.

Varios módulos del sistema se han virtualizado con docker, lo que permite un despliegue sencillo en cualquier servicio de hosting con soporte de contenedores, como es Amazon Web Services.

\figura{1}{img/disenio_sistema/docker_vs_vm.png}{Docker VS VM}{dockervsvm}{}
\pagebreak

\section{Productores de datos}\label{sec:productores}
El m'odulo de producci'on de datos es el encargado de recolectar la informaci'on necesaria para el funcionamiento del sistema.

En este caso, los datos necesarios son los valores del \textbf{EUR/USD} y los valores del \textbf{BTC/USD}.
En el mercado existen diversas herramientas para obtener la información, pero éstas han sido descartadas, bien por ser herramientas de pago o debido a que la calidad y precisión del dato no es la suficiente. Finalmente, después de sopesar las opciones disponibles, se ha decidido obtener la información de la página \url{https://es.investing.com/} mediante \textbf{web scraping}. 

\subsection{Extracción del dato}
Para conseguir la información de las divisas desde la fuente de datos necesitamos primero descargar el fichero HTML resultante de realizar una petición HTTP a las URLs: 
\begin{itemize}
\item EUR/USD - \url{https://es.investing.com/currencies/eur-usd}
\item BTC/USD - \url{https://es.investing.com/crypto/bitcoin/btc-usd}
\end{itemize}


Una vez descargados los ficheros es necesario un estudio de los mismos, para detectar así que elementos HTML contienen la información deseada. En este caso, el valor de la divisa se encuentra dentro de un elemento span con identificador \textit{last\textunderscore last}. Debido a que la estructura de los ficheros es idéntica ha sido posible generalizar la extracción de información para cualquier divisa de la página web, por lo que el número de divisas con las que opera el sistema es altamente escalable.

\subsection{Envío de la información}
Una vez se ha obtenido el valor de la divisa éste necesita ser enviado al módulo siguiente. Para gestionar todos los mensajes generados por los productores de datos se necesita un gestor de cola de mensajes, el cual actúa de intermediario entre los productores de datos y el resto del sistema, asegurando que los mensajes enviados llegan a los componentes que los necesiten. 
Actualmente el flujo de información generado por los productores es ínfimo pero una vez se amplíe el número de divisas y el número de fuentes de información se puede presentar una situación en la que se produzcan problemas de congestión y perdida de mensajes.

\subsection{Repetición}
Para obtener un flujo continuo de información se necesita que los dos pasos explicados anteriormente se repitan en el tiempo a intervalos regulares. Debido a la velocidad de las operaciones del mercado Forex se ha configurado la extracción de información cada minuto.

\subsection{Tecnologías utilizadas}
Para implementar las distintas partes de este módulo se ha hecho uso de las siguientes tecnologías
\begin{itemize}
\item BeautifulSoup
\item Requests
\item Confluent-kafka
\item Schedule
\item Docker
\end{itemize}
\pagebreak
\subsubsection*{BeautifulSoup}
Beautifulsoup es una librería de Python para el análisis y extracción de datos en ficheros HTML.
Ésta crea una estructura en árbol que puede ser utilizada para la extracción de información de ficheros HTML.

\codigofuente{Python}{B'usqueda de elementos span}{codigo/soup.py}

\subsubsection*{Requests}
Requests es una librería Python que permite realizar peticiones HTTP,  pensada para ser usada por humanos, por lo que presenta una sintaxis sencilla y legible. La librería soporta desde los métodos HTTP básicos, como POST o GET, hasta gestión de cookies, SSL, proxy y streaming.

Esta librería ha sido utilizada para descargar los ficheros HTML que después serán procesados por beautifulsoup.

\subsubsection*{Confluent-kafka}
Confluent-kafka es la librería que nos provee de los métodos necesarios para conectarnos al componente Apache kafka del siguiente módulo. Esta librería en necesaria, ya que registra al scraper como un producer de kafka y permite que la información enviada para ellos se gestione y se distribuya adecuadamente.

\subsubsection*{Schedule}
Schedule es una librería Python para la programación de tareas periodicas, permitiendo ejecutar funciones en periodos personalizables.

\subsubsection*{Docker}
Se ha creado una imagen de docker genérica que implementa las tareas necesarias para la extracción de los datos. Para inicializar la imagen solo es necesario proporcionar el nombre de la divisa que se quiere obtener. 
\clearpage




\section{Procesado y almacenamiento}\label{sec:procesado}
El módulo de procesamiento y almacenamiento de datos se ocupa de recoger la informaci'on generada por el resto de módulos, tratarla, almacenarla y enviarla a los m'odulos que la necesiten.

Las tareas que desempeña este módulo son:
\begin{itemize}
\item Recepción.
\item Tratamiento.
\item Almacenamiento.
\end{itemize}

\subsection{Recepción}
El módulo recibe directamente la información en bruto de los productores. Para que no se produzca una saturación entre mensajes y puedan perderse se necesita un gestor de colas de mensajes. Este gestor recibe los mensajes de los productores y los almacena temporalmente, agrupándolos por topics previamente configurados, hasta que son demandados, momento en el cual el gestor envía los mensajes hasta los demandantes.

Este proceso se realiza en Apache Kafka.

\subsection{Tratamiento}
Una vez obtenida la información es necesario tratarla para poder alimentar la red neuronal con ella. La información de cada divisa se encuentran en grupos de quince minutos, y pasa un proceso de procesado que consta de los siguientes pasos:
\begin{enumerate}
\item \textbf{Obtención del dato}. Se elimina toda la información no necesaria, como posibles letras, guiones o espacios, y se almacena en los diferentes campos de la fila del dataset. 
\item \textbf{Casteo a número real}. Los datos numéricos que nos llegan cambian de tipo para poder realizar operaciones matemáticas de coma flotante.
\item \textbf{Casteo de fechas}. Los datos en forma de fecha que llegan al sistema se castean a un tipo Datetime, que consta de fecha y hora, para poder utilizarlo fácilmente a la hora de hacer b'usquedas.
\item \textbf{Ordenado de dataframe}. Como medida de seguridad, se ordena el dataset de forma creciente tomando como criterio la fecha.
\item \textbf{Retorno logarítmico}. Se calcula el retorno logarítmico entre los valores que han llegado, para facilitarle los c'alculos al módulo de la red neuronal.
\item \textbf{Almacenamiento}. Como paso final, se almacenan los datos en la base de datos.

\end{enumerate}

Este proceso se realiza en Apache Spark.



\subsection{Almacenamiento}
Con la información ya agrupada y tratada se procede a almacenarla en un sistema de almacenamiento a largo plazo donde el resto de módulos puedan consultarlas. 

Este proceso se realiza en MySQL

\subsubsection{Estructura de la base de datos}
Las tablas de la base de datos se separan en tres grupos: predicciones, tiempo real e histórico.

Las tablas de predicciones almacenan los resultados generados por las redes neuronales. Hay cuatro tablas de este tipo, una por cada par red/divisa.

Las tablas de tiempo real almacenan los datos actuales hasta un m'aximo de quince en el pasado. Por tanto, si el sistema se encuentra en el instante t10 esta tabla contará con los valores del momento actual y los 9 instantes pasados, pero al llegar al instante t16 la tabla vuelca sus datos en las tablas de histórico, conservando 'unicamente el valor del instante actual. Existe una tabla de este tipo por divisa

En las tablas de histórico se guardan los datos pasados de las divisas. Existe una tabla por divisa.



\subsection{Tecnologías utilizadas}
Por cada tarea a cumplir se han desplegado tres sistemas que cumplen las exigencias necesarias. Estos sistemas son:
\begin{itemize}
\item Apache Kafka
\item Apache Spark
\item Base de datos MySQL
\item Docker
\end{itemize}

Los componentes Apache Spark y Apache Kafka se explicarán en más detalle en el capítulo \ref{cap5}.

\subsubsection{MySQL}
Para el almacenamiento a largo plazo de los datos se ha optado por una base de datos MySQL.

MySQL es un sistema de gestión de bases de datos relacional, desarrollado por Oracle Corporation bajo una licencia dual pública general/comercial.
MySQL está desarrollado en C y C++ y destaca por su gran adaptación a diferentes entornos, permitiendo su integración con los lenguajes de programación más utilizados como Java y Python.

\subsubsection{Docker}
Se ha utilizado Docker para virtualizar tanto el Apache Kafka como la base de datos MySQL. El hecho de virtualizar Apache Kafka permite desplegar instancias del mísmo a demanda con relativa facilidad.

\clearpage


\section{Red neuronal}\label{sec:red}
Las redes neuronales son el pilar fundamental del presente proyecto, ya que son el sistema encargado de predecir nuevos movimientos burs'atiles, en base a un hist'orico conocido. 

En este proyecto se van a desarrollar dos arquitecturas diferentes de redes neuronales, siguiendo una de ellas la arquitectura \textbf{recurrente} y la otra la arquitectura \textbf{convolucional}.



\subsection{Consideraciones previas}

Antes de explicar la arquitectura de las redes neuronales, debemos entender dos conceptos fundamentales. 

Predecir movimientos burs'atiles es un problema de regresi'on aplicado a series temporales. En dichas series suelen confundirse los conceptos de pasado, presente y futuro. Cuando se habla de \textbf{presente} no se est'a hablando del presente actual, ni del momento en el que se lee este p'arrafo sino al \textbf{momento temporal de referencia}, que por convenci'on se le llama \textbf{t0}.
Expresado de otro modo, cada vez que vamos a realizar una operaci'on sobre una serie temporal se debe tomar un momento de referencia y una vez se selecciona debemos imaginar el contexto del problema como si el tiempo se parase. Es decir, si se dice que se van a realizar operaciones durante un momento temporal, el tiempo que se tarda en realizar esas operaciones \textbf{no se tiene en cuenta}.




\subsubsection*{Vector de entrada}
El vector de entrada de la red tiene una longitud temporal de \textbf{tres horas}, o lo que es lo mismo \textbf{ciento ochenta minutos}. Este vector contiene informaci'on del valor de la moneda 3 horas antes a t0.

Para ejemplificarlo, consideramos t0 el d'ia de hoy las 18:00. Una vez elegido t0, se crea una secuencia con todos los valores de la moneda desde t-180 hasta t-1, es decir, todos los valores de la moneda por minuto desde las 15:00 hasta las 17:59.
Dicho ejemplo se presenta en la figura \ref{vectorentrada}.

\figura{1}{img/disenio_sistema/vector_entrada.png}{Ejemplo del vector de entrada con un momento temporal concreto}{vectorentrada}{}

Esta secuencia se convierte en un vector, y as'i creamos \textbf{vector de entrada}. 


\subsubsection*{Vector resultado}
El vector resultado de la red contiene la informaci'on de la predicci'on que ha hecho la red neuronal. Este vector tiene una longitud temporal de cuarenta y cinco \textbf{minutos}.


Siguiendo el mismo ejemplo que en el vector de entrada, este vector resultado contiene los valores de la moneda desde t0 hasta t44, lo que genera el valor de la moneda en todos los minutos desde las 18:00 hasta las 18:45. 
Dicho ejemplo se presenta en la figura \ref{vectorsalida}.

\clearpage

\figura{1}{img/disenio_sistema/vector_salida.png}{Ejemplo del vector resultado con un momento temporal concreto}{vectorsalida}{}

Este vector es el resultado de los c'alculos que ha realizado la red neuronal. Es importante recordar que los datos vienen ordenados, por lo que una vez obtenido el vector resultado se pueden visualizarse directamente. 



\subsubsection*{Early Stopping}
El Early Stopping es una herramienta que proporciona Tensorflow para detener el entrenamiento de la red neuronal en el caso de que 'esta deje de aprender. Esta herramienta se utiliza para evitar el sobreajuste durante el entrenamiento.

\figura{0.7}{img/disenio_sistema/earlystopping.png}{Gr'afico de la precisi'on en el aprendizaje indicando la 'epoca 'optima para parar el aprendizaje}{earlystopping}{}

Como se puede apreciar en la figura \ref{earlystopping}, la precisi'on empieza a bajar pasado un n'umero de epocas debido a que est'a sobreajust'andose a los valores que ya conoce. La l'inea roja representa la precisi'on de las predicciones de los datos que está utilizando para entrenar, mientras que la linea azul representa la precisi'on en las predicciones de unos datos que la red no está usando para entrenar, por lo que no conoce.  

El early stopping tiene varios par'ametros de configuraci'on. Podemos destacar el \textbf{valor delta}, el cual indica el valor que debe \textbf{mejorar} la m'etrica de error para considerar que sigue aprendiendo y la \textbf{paciencia}, el cual indica el n'umero de 'epocas que permite a la red seguir entrenando sin mejorar su delta. 

\clearpage

\subsection{Red Neuronal Recurrente}

Las redes neuronales recurrentes proporcionan muy buenos resultados en problemas de \textbf{series temporales}. Como se ha podido observar en el cap'itulo \ref{cap2}, la mejor capa que podemos utilizar en este tipo de redes es la capa \textbf{LSTM}.

El algoritmo que se ha construido es el siguiente:

\figura{0.3}{img/disenio_sistema/SECUENCIAL.png}{Arquitectura de red neuronal recurrente utilizada en el proyecto}{secuencial}{}

En este esquema se puede observar cómo se han implementado dos capas LSTM, la primera de 180 neuronas y la segunda de 128 neuronas adem'as de una capa Dense de 45 neuronas que genera la salida de la red neuronal. Las capas LSTM tienen asociadas unas capas Dropout que no se ven en el diagrama. Este dropout elimina el 10\% de las neuronas, escogidas aleatoriamente.

\subsubsection*{Funcionamiento}

Para cada vector que pertenece a la lista de vectores de entrada el proceso que se le realiza es el siguiente:

\begin{enumerate}
\item El vector entrada entra por la capa de input teniendo una longitud de 180 neuronas.
\item El vector entra en la primera capa LSTM, donde se procesa secuencialmente. El resultado de esta capa es un vector de dimensi'on 128. 
\item El vector que ha resultado del paso 2 entra por la segunda capa LSTM, proces'andose secuencialmente. El resultado de esta capa es una lista de 128 vectores de dimensi'on 1, es decir, un vector por neurona.
\item La capa dense recibe todos los vectores y los interconecta entre s'i con 45 neuronas que van a generar el vector resultado.
\item Desde la capa densa sale un vector de longitud 45 el cual es el vector salida.
\end{enumerate}


\subsubsection*{Entrenamiento}

La red neuronal recurrente entrena durante un periodo de ocho d'ias de anterioridad respecto al momento t0.
Los datos obtenidos se transforman en secuencias de 3 horas desde las 00:00 del octavo d'ia anterior, sin contar los d'ias en los que el mercado cierra. 

Se ha seleccionado el periodo de 8 días debido a que es un periodo lo suficientemente largo (se crean alrededor de 10000 secuencias) para que la red pueda encontrar patrones sin sobreajustarse. Hay que tener en cuenta que los valores oscilan en magnitudes muy pequeñas. Debido a esto hay que normalizar los datos entre 0 y 1, consiguiendo así que la red pueda realizar cálculos con números de mayor magnitud.  

La red tiene un Dropout muy bajo debido a que los patrones que encuentra no son patrones muy generalizados. Por ello, al poner una tasa de Dropout mayor la red es incapaz de encontrar un patr'on m'inimamente v'alido. 

La red es sometida a 150 'epocas de entrenamiento. Durante estas 150 épocas, la red va a ver los datos con el fin de encontrar patrones nuevos cada vez que los revisa, o reforzar el conocimiento que tiene sobre un patr'on. 
La red tiene configurado un EarlyStopping de 10 épocas con un delta de 0.00001 de mejora en su m'etrica de evaluaci'on. Por norma general, el EarlyStopping salta sobre la época 110. 


\subsubsection*{Benchmark}

La red ha sido sometida a muchos benchmarks para evaluar su rendimiento, tanto en tiempo de procesamiento como en calidad de la predicci'on. 

En la figura \ref{benchmarklstm} se puede observar un ejemplo de una salida t'ipica de la red usando el par EUR/USD. 

\figura{1}{img/disenio_sistema/recurrent_benchmark.png}{Resultado t'ipico de la predicción de la red neuronal recurrente}{benchmarklstm}{}

La l'inea azul representada en la leyenda como \textit{train}, representa los 180 valores que ha utilizado la red para predecir la l'inea verde, representada en la leyenda como \textit{prediction}. La l'inea naranja representa los valores reales que ha tomado el par en el mismo periodo que ha predicho la red. 

Lo primero que se observa es que no tiene una precisi'on total, y as'i es. Como ya hemos explicado en la secci'on de fundamentos teóricos, esta red se est'a basando exclusivamente en el hist'orico del valor, sin recibir ninguna informaci'on extra, por lo que una bajada brusca que se deba a un factor externo le resulta muy difícil predecirla.

No obstante, la red est'a prediciendo fácilmente la tendencia que est'a siguiendo el valor de la moneda. Adem'as, en la l'inea naranja, se est'a mostrando el valor al que se cierra el par en ese minuto. 'Esto no quiere decir, que el valor que la red ha predicho no haya ocurrido, sino que durante ese minuto ese valor ha oscilado entre varios. Suponiendo que cambia su valor cada segundo, habrá oscilado 60 veces entre dos puntos de la gr'afica. Existe una probabilidad muy alta de que el valor que la red ha predicho, aunque no haya sido el valor final, sea un valor que el par ha tomado durante ese minuto.  




\clearpage




\subsection{Red Neuronal Convolucional}

Las redes neuronales convolucionales originalmente no est'an diseñadas para ser utilizadas en problemas de series temporales, no obstante, éstas obtienen patrones basándose en los detalles de los datos. Si extrapolamos el problema, haciéndole ver a la red que lo que recibe es un \textbf{mapa} de datos en vez de una secuencia, la red puede funcionar correctamente y predecir resultados. Este mapa tiene \textbf{la misma dimensi'on} que la secuencia, pero la red convolucional no va a darle importancia al orden en el que recibe los datos, tal y como hace la red recurrente. 
Al necesitar que los datos est'en ordenados obligatoriamente en la predicci'on, le indicaremos a la red que trate todos los mapas que va a procesar separándolos por conjuntos independientes. De esta forma la red sabe que cada secuencia, las cuales ahora son un mapa, deben ser tratadas y procesadas de forma individual. 
De esta manera la red va a ir aprendiendo y entendiendo la informaci'on que proporciona cada uno, adaptando sus pesos para conseguir una soluci'on para todos lo mapas que va a recibir.


El algoritmo que se ha construido es el siguiente:

\figura{0.2}{img/disenio_sistema/CONV.png}{Arquitectura de red neuronal convolucional usada en el proyecto.}{convolucional}{}

En el anterior esquema se puede apreciar cómo que se han implementado dos capas Conv1D, la primera con 128 neuronas, y la segunda de 64 neuronas, encontrándose ambas concatenadas con capas MaxPooling1D. A 'esto se le añade una capa Flatten y una capa Dense de 45 neuronas, lo que genera la salida de la red neuronal.

\clearpage

\subsubsection*{Funcionamiento}

Para cada vector que aparece en la lista de vectores de entrada, el proceso que se realiza en la red es el siguiente:

\begin{enumerate}
\item El vector entrada entra por la capa de input, teniendo una longitud de 180.
\item El vector entra en la primera capa Conv1D, lo que hace que se reduzca  la dimensionalidad de esta capa a 64. El resultado de esta capa es un vector de dimensi'on 64. 
\item El vector que ha resultado del paso 2 entra por la capa MaxPooling1D, reduciendo su dimensionalidad a la mitad. 
\item Tras el paso 3, los datos vuelven a pasar por una capa Conv1D, siendo el resultado de esta capa un vector de dimensi'on 32.
\item El vector resultante pasa por la capa MaxPooling1D, reduciendo su dimensi'on a la mitad.
\item El vector entra en una capa Flatten, donde los datos se concatenan en una lista. 
\item El vector resultante del paso 7 entra en una capa Dense, en la cual se van a interconectar entre s'i las neuronas de la capa Flatten con 45 neuronas, generando as'i el vector resultante.
\item De la capa Dense sale un vector de longitud 45. Este es el vector salida.
\end{enumerate}





\subsubsection*{Entrenamiento}

La red neuronal convolucional entrena durante un periodo de diez días de anterioridad respecto al momento t0.
Estos datos se transforman en secuencias de 3 horas desde las 00:00 del d'ecimo d'ia anterior excluyendo los días en los que los mercados cierran. 

Se ha escogido la cifra de 10 d'ias frente a los 8 d'ias de la red recurrente porque esta red necesita un volumen mayor de datos para aprender, pero, al igual que en la red recurrente, si el periodo es muy largo la red se sobreajustar'a basándose en los datos que conoce. 

La red es sometida a 150 'epocas de entrenamiento, al igual que las redes recurrentes. Tiene configurado un EarlyStopping de 10 'epocas con un delta de 0.00001 de mejora en su m'etrica de evaluaci'on. Por norma general, el EarlyStopping salta, aproximadamente, en la 'epoca 64. 



\subsubsection*{Benchmark}

\figura{1}{img/disenio_sistema/convolutional_benchmark.png}{Resultado t'ipico de la prediccion de la red neuronal convolucional}{benchmarkconv}{}


Al igual que sucede con la red recurrente, la red convolucional es sometida a muchos benchmarks para evaluar su rendimiento. 

En la figura \ref{benchmarkconv} se presenta un ejemplo de una salida t'ipica de la red usando el par EUR/USD, con unos datos similares al benchmark de la red recurrente. 


La l'inea azul representada en la leyenda como \textit{train} representa los 180 valores que ha utilizado la red para predecir la l'inea verde, representada en la leyenda como \textit{prediction}. La línea naranja representa los valores reales que ha tomado el par en el mismo periodo que ha predicho la red. 

A diferencia de la red recurrente, las predicciones de esta red parecen m'as volátiles o err'aticas. Esto es debido a la magnitud de los datos, la cual es muy pequeña, aunque los datos se normalicen, resulta difícil seguir la predicci'on con una precisi'on total. Esta red tambi'en es capaz de seguir la tendencia de los valores, pero sus predicciones son menos fiables que las que proporciona la red recurrente. 

\subsection{Tecnologías utilizadas}
Para la implentación de ambas redes neuronales se ha utilizado TensorFlow que se explicará en profundidad en el capítulo \ref{cap5}.


\clearpage



\section{Monitorizaci'on}\label{sec:monitorizacion}
El módulo de monitorizaci'on se encarga de la visualizaci'on y representaci'on de los datos generados por el sistema, permitiendo al usuario interpretarlos a través de una interfaz de usuario web.
Los valores generados por el sistema son las predicciones de las divisas hechas por las redes neuronales. Por cada divisa el sistema proporciona dos predicciones, cada una generada por una red neuronal diferente.

Lo primero que muestra la interfaz es el valor actual de las divisas y su cambio respecto al valor anterior, mostrándose de color verde, rojo o blanco, dependiendo de si el valor es mayor, menor o igual al valor anterior, respectivamente. Todo ello permite al usuario de forma rápida y sencilla conocer la situación actual de las divisas. En la figura \ref{dash1} se puede observar un ejemplo de la misma.

\figura{1}{img/disenio_sistema/dash1.png}{Valores actuales de las divisas}{dash1}{}

A continuación, se encuentra la gráfica principal, en la cual se muestran las predicciones de los próximos cuarenta y cinco minutos, el valor actual de la divisa y los últimos treinta minutos del histórico de valores, tal y como se puede observar en la figura \ref{dash2}.
Mediante un menú desplegable podemos seleccionar la divisa que queremos visualizar y, a través de las pestañas se puede alternar entre las dos redes neuronales.

\figura{1}{img/disenio_sistema/dash2.png}{Representaci'on de los resultados}{dash2}{}

Finalmente, el módulo presenta una gráfica con el histórico completo de la divisa seleccionada. En la figura \ref{dash3} se puede observar un ejemplo con el histórico del BTC/USD.
\figura{1}{img/disenio_sistema/dash3.png}{Hist'orico del BTC/USD}{dash3}{}

En esta gráfica podemos elegir mediante el menú desplegable de la izquierda el tipo el gráfico que queremos representar y, mediante el menú desplegable de la derecha podemos agregar indicadores al gráfico. En la figura \ref{dash4} se puede observar un ejemplo con el histórico del BTC/USD junto a los indicadores MA y EMA.

\figura{1}{img/disenio_sistema/dash4.png}{Hist'orico del BTC/USD con indicadores}{dash4}{}

Los indicadores de los que dispone el sistema son:
\begin{itemize}
\item Acumulación distribución
\item Bandas de Bollinger
\item Media móvil
\item Media móvil exponencial
\item Índice de canales de productos básicos
\item Tasa de cambio
\item Puntos pivote
\item Oscilador estocástico
\item Momentum
\end{itemize}

\subsection{Tecnologías utilizadas}
Para la visualización y representación de los datos se ha escogido el framework Dash.

\subsubsection{Dash}
Dash es un framework de Python destinado a la creaci'on de aplicaciones web de código abierto, desarrollado por Plotly.
Dash está implementado sobre Flask, Plotly.js y React.js, especializándose  en la creación de aplicaciones reactivas de visualizaci'on de datos con interfaces de usuario altamente personalizables en Python, abstrayendo todas las tecnolog'ias y protocolos necesarios para construir una aplicaci'on interactiva basada en la web.
El c'odigo de la aplicaci'on Dash es declarativo y reactivo, lo que facilita la creaci'on de aplicaciones complejas que contienen muchos elementos interactivos, siendo cada elemento est'etico de la aplicaci'on completamente personalizable (tamaño, color, posici'on, fuente, ...)
Las aplicaciones de Dash son desplegadas mediante Flask y se comunican mediante JSON sobre HTTP. 
La interfaz de Dash presenta componentes utilizando React.js, la biblioteca de interfaz de usuario de Javascript escrita y mantenida por Facebook.
Los componentes de Dash son clases de Python que codifican los valores y propiedades de los componentes de React.js, siendo éstos serializados en JSON, aunque también existen clases Python para todos los componentes HTML y sus propiedades, por lo que mediante c'odigo Python se pueden declarar y manipular todos los elementos de una interfaz y sus interacciones entre ellos.





\pagebreak

