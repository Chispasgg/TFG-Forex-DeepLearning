\chapter{Conclusiones}\label{conclusiones}

Para concluir la memoria de este proyecto, se va a realizar un repaso de lo aprendido durante el proceso. El objetivo de este proyecto ha sido desarrollar una arquitectura software con Big Data para predecir valores del mercado de valores, aplicando técnicas de Deep Learning. 

En la primera fase del desarrollo del proyecto se va a realizar un estudio y análisis de las diferentes técnicas Big Data, algoritmos de Deep Learning, estudio de la ciencia de datos, así como del mercado de valores.

Con el propósito de realizar un proyecto innovador y revolucionario, se ha desarrollado un algoritmo capaz de interpretar los valores de un mercado financiero y realizar predicciones en base a dichos valores. Además, se ha desarrollado todo un ecosistema software, apoyado con la herramienta software más potente del mercado para entornos Big Data. Para visualizar e interpretar los resultados, se ha realizado un panel de visualización con el que el usuario puede estudiar el mercado de forma sencilla e intuitiva. 



Respecto al resultado del proyecto se puede concluir que, a pesar de la potencia con la que est'an dotadas las redes neuronales, entrenar los modelos 'unicamente con los valores del hist'orico de una divisa no hace posible que se generen unas predicciones con la precisi'on necesaria para confiar nuestros activos a un sistema que opere de manera aut'onoma.

Se han probado arquitecturas muy complejas durante la fase de desarrollo, algunas de ellas tardaban un día en entrenar el modelo por completo, pero el resultado es muy similar a la arquitectura finalmente propuesta, la cual se ha determinado como un equilibrio entre precisión y rendimiento.

Trabajar con tecnolog'ias tan novedosas como TensorFlow o Spark fue costoso al inicio del proyecto. Debido a su curva de aprendizaje se ha necesitado mucho tiempo de investigaci'on y pr'actica hasta conseguir el nivel de destreza necesaria para desenvolvernos con dichas tecnologías. 
No obstante, ha sido una experiencia enriquecedora en la que hemos disfrutado bastante durante el aprendizaje y el desarrollo, que nos ha aportado mucho conocimiento que ayudar'a a seguir form'andonos y a progresar en el campo de la inteligencia artificial y la ciencia del dato.


	


