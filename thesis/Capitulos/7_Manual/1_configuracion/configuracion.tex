Antes de poder utilizar el sistema, hay que tener en cuenta unas consideraciones previas.

\subsection{Requisitos m'inimos}

\begin{itemize}
\item CPU: CPU con 4 núcleos o superior.
\item RAM: 8 GB o superior
\item SO: Ubuntu 14.04 o superior
\item Software: El sistema debe tener instalado Python en su versi'on 3.5 o superior, junto a la herramienta Pip3.
\end{itemize}

\subsection{Entorno}

En la carpeta \emph{tools} del c'odigo del proyecto se encuentra un archivo llamado \textit{requirements.txt}, que contiene las dependencias necesarias para ejecutar el proyecto. Para instalarlas basta con ejecutar en una consola bash de Ubuntu este comando:


\begin{lstlisting}
pip3 install -r requirements.txt
\end{lstlisting}
\addcontentsline{lol}{lstlisting}{\protect\numberline{\thelstlisting}Instalaci'on de requisitos}

\clearpage

Una vez instaladas las dependencias de Python, hay que instalar Spark. 

En la carpeta \emph{tools/spark} se encuentra un documento redactando siguiendo la sintaxis Markdown en el que se encuentran una guía de instalación que hemos diseñado para que se pueda instalar Spark de forma sencilla.

Además, hay que instalar Docker y Docker Compose. Para ello, es necesario abrir una consola bash y lanzar los siguientes comandos :

\begin{lstlisting}
sudo apt-get update
sudo apt install docker.io
\end{lstlisting}
\addcontentsline{lol}{lstlisting}{\protect\numberline{\thelstlisting}Instalaci'on de docker}

Una vez instalado, se inicializa el servicio con el comando:

\begin{lstlisting}
sudo systemctl start docker
\end{lstlisting}
\addcontentsline{lol}{lstlisting}{\protect\numberline{\thelstlisting}Inicializaci'on de docker}

Posteriormente, hay que ejecutar estos comandos para instalar Docker Compose:

\begin{lstlisting}
sudo curl -L "https://github.com/docker/compose/releases/download/1.23.1/docker-compose-$(uname -s)-$(uname -m)" -o /usr/local/bin/docker-compose
sudo chmod +x /usr/local/bin/docker-compose
\end{lstlisting}
\addcontentsline{lol}{lstlisting}{\protect\numberline{\thelstlisting}Instalaci'on de docker-compose}


La instalaci'on se verifica lanzando este comando:
\begin{lstlisting}
docker-compose --version
\end{lstlisting}
\addcontentsline{lol}{lstlisting}{\protect\numberline{\thelstlisting}Verificaci'on de la instalaci'on}

Si todo ha ido bien, devolver'a por pantalla el siguiente mensaje:
\begin{center}
\textit{docker-compose version 1.23.1, build b02f1306}
\end{center}






